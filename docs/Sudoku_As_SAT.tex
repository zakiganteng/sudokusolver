% vim: set fenc=utf-8 ft=latex encoding=utf-8
% -*- mode: latex; coding: UTF-8; -*-
\documentclass[conference,final]{IEEEtran}

\usepackage[T1]{fontenc}
\usepackage[utf8]{inputenc}


\newcommand{\TheTitle}{Investigating the Effects of Extended Encoding Against
Minimal Encoding}
\newcommand{\TheKeywords}{Sudoku, SAT-solver, minimal encoding, extended encoding}
\newcommand{\TheAuthors}{Evan Wilde, Sebastien Guillemot, Shayla Redlin, Laura
Grondahl}
\newcommand{\TheSubject}{Solving sudoku of varying size and difficulty with a
SAT solver}
\usepackage[hyphens]{url}
\urlstyle{same}
\usepackage[unicode=true,
            bookmarks=false,breaklinks=false,pdfborder={0 0 0},
            backref=none,colorlinks=false]{hyperref}



\hypersetup{pdftitle={\TheTitle},
            pdfauthor={\TheAuthors},
            pdfkeywords={\TheKeywords},
            pdfsubject={\TheSubject},
            urlcolor=blue,citecolor=red}

\usepackage[nospace]{cite}


% Fix any bad-hyphenations here
\hyphenation{}

\begin{document}

\title{\TheTitle}

\author{
\IEEEauthorblockA{\TheAuthors}
\IEEEauthorblockN{Department of Computer Science,
                    University of Victoria,
	    Canada.}}
\maketitle
\begin{abstract}
In this paper we explore solving sudoku as a SAT problem. Using encoding
methods described by Ines Lynce and Joel Ouaknine, we wrote a program to take
an unsolved sudoku puzzle, encode it in standard SAT-challenge (DIMACS) format,
run the SAT solver, and finally convert the results back to a sudoku board.
Lynce and Ouaknine identify a minimal encoding and an extended encoding in
their paper. We implemented both encodings in separate programs. The program is
capable of running on NxN puzzles where N is a perfect square. e.g 9x9, 16x6,
and 25x25 are all usable board sizes. Due to base conversions, our program is
only capable of going up to board sizes of 36x36. The paper presents the
runtime of the minimal encoding versus the extended encoding in the program
``minisat''. We also present the method by which our program was generalized to
handle puzzles of varying sizes, and how varying the size changes the runtime.
\end{abstract}
\begin{IEEEkeywords}
\TheKeywords
\end{IEEEkeywords}

\section{Introduction}
In this paper we explore solving sudoku as a SAT problem. Using encoding
methods described by Ines Lynce and Joel Ouaknine, we wrote a program to take
an unsolved sudoku puzzle, encode it in standard SAT-challenge (DIMACS) format,
run the SAT solver, and finally convert the results back to a sudoku board.
Lynce and Ouaknine identify a minimal encoding and an extended encoding in
their paper. We implemented both encodings in separate programs. The program is
capable of running on NxN puzzles where N is a perfect square. e.g 9x9, 16x6,
and 25x25 are all usable board sizes. Due to base conversions, our program is
only capable of going up to board sizes of 36x36. The paper presents the
runtime of the minimal encoding versus the extended encoding in the program
``minisat''. We also present the method by which our program was generalized to
handle puzzles of varying sizes, and how varying the size changes the runtime.

\section{Data sample}
We found an online board
generator\footnote{\url{http://sudoku-puzzles.merschat.com/}} with difficulties
ranging from ``easy'' to ``Lex Luthor''. Easier puzzles have many blocks
originally filled in, whereas puzzles at Lex Luthor difficulty only have a few
positions filled in, adding more potentially ambiguous cases. On a 9x9 board,
an ``easy'' board may have more than 50\% of the squares filled in, whereas a
``Lex Luthor'' board may have as few as 4 squares filled in.



\section{Conclusion}


\end{document}
