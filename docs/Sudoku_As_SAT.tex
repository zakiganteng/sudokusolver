% vim: set fenc=utf-8 ft=latex encoding=utf-8
% -*- mode: latex; coding: UTF-8; -*-

\newif\ifdraft
% \drafttrue % Draft mode -- comment this out before submission

\ifdraft
  \documentclass[conference,draftclsnofoot]{IEEEtran}
\else
  \documentclass[conference,final]{IEEEtran}
\fi
\def\baselinestretch{1}
\setlength{\marginparwidth}{2cm}

\usepackage[T1]{fontenc}
\usepackage[utf8]{inputenc}


\newcommand{\TheTitle}{Investigating the Effects of Extended Encoding Against
Minimal Encoding}
\newcommand{\TheKeywords}{Sudoku, SAT-solver, minimal encoding, extended encoding}
\newcommand{\TheAuthors}{Evan Wilde, Sebastien Guillemot, Shayla Redlin, Laura
Grondahl}
\newcommand{\TheSubject}{Solving sudoku of varying size and difficulty with a
SAT solver}
\usepackage[hyphens]{url}
\urlstyle{same}
\usepackage[unicode=true,
            bookmarks=false,breaklinks=false,pdfborder={0 0 0},
            backref=none,colorlinks=false]{hyperref}

\hypersetup{pdftitle={\TheTitle},
            pdfauthor={\TheAuthors},
            pdfkeywords={\TheKeywords},
            pdfsubject={\TheSubject},
            urlcolor=blue,citecolor=red}

\usepackage[nospace]{cite}

% Table support
\usepackage{dcolumn}
\usepackage{longtable}
\usepackage{balance}
\usepackage{placeins}
\usepackage{multirow}

% Extra suppport
\usepackage{xspace}
\usepackage{caption}

% Fix any bad-hyphenations here
\hyphenation{puzzles extended minimal encodings minisat SAT-solver levels
difficulty}

\ifdraft
	\usepackage[colorinlistoftodos]{todonotes}
	\newcommand{\evan}[1]{{\color{blue}\emph{Evan says: #1}}\xspace}
	\newcommand{\evantodo}[1]{{\color{blue}\emph{Evan Todo: #1}}\xspace}
\else
	\usepackage[disable]{todonotes}
	\newcommand{\evan}[1]{}
	\newcommand{\evantodo}[1]{}
\fi


\begin{document}

\title{\TheTitle}

\author{
\IEEEauthorblockA{\TheAuthors}
\IEEEauthorblockN{Department of Computer Science,
                    University of Victoria,
	    Canada.}}
\maketitle
\begin{abstract}
 The Sudoku game was popular in Japan since the mid-80's, but in 2005 had an
 explosion in western countries. The popular game can be encoded as a SAT
 problem and given to a SAT solver to get a solution to the puzzle. Various
 encodings of the Sudoku problem have been presented.	In this paper we
 explore the scalability of extended encoding and minimal encoding of Sudoku as
 SAT over puzzle difficulty levels and board sizes.
 \end{abstract}
\begin{IEEEkeywords}
\TheKeywords
\end{IEEEkeywords}

\section{Introduction}
The game of Sudoku became popular in Japan in the mid-80's\cite{Lynce2006}, but
in 2005 had an explosion in western countries. The popular game can be
encoded as a SAT problem and given to a SAT solver to get a solution to the
puzzle. Various encodings of the Sudoku problem have been presented.	In this
paper we explore the scalability of extended encoding and minimal encoding of
Sudoku as SAT over puzzle difficulty levels and board sizes. We ran the
encodings using the ``minisat''\footnote{\url{http://minisat.se/MiniSat.html}}
SAT solver to get our results. All of our results are available on a GitHub
repository\footnote{\url{https://github.com/etcwilde/sudokusolver}}.


\section{Related Work}
There has been research in the area of finding cpu and memory efficient
encodings of sudoku as a SAT problem\cite{Lynce2006, Kwon2005}. Both papers
look at a way to get better performance from a SAT solver on the sudoku
problem. \textit{Sudoku as a SAT Problem} simply presents an extended encoding
which generates more rules for the board, which limit the number of ambiguities
that the SAT solver must attempt to solve, thus increasing the CPU performance.
The full extended encoding is a $O(n^4)$ algoroithm, so as the board size
grows, the feasibility of generating the necessary clauses drops.
\textit{Optimized CNF Encoding for Sudoku Puzzles} looks at obtaining good
results from the SAT solver, but also good results from the encoding process
itself\cite{Kwon2005}, decreasing the cpu time, memory requirements,and the
number of clauses to be generated. Our paper does not look into the efficient
encoding, only extended and minimal encodings.

\section{Methodologies}
To test the benefits and drawbacks of the presented minimal and extended
encodings, we performed the following tests to determine how each encoding
scales across the difficulty of the puzzle and the size of the puzzle. The
puzzles we used for testing purposes are available in the GitHub repository.\\
\begin{enumerate}
	\item Minimal encoding on $9 \times 9$ puzzles of ``easy'', ``medium'', and
		``Lex Luthor'' difficulty levels.
	\item Minimal encoding on unsolvable $9 \times 9$ puzzles of ``easy'',
		``medium'' and ``Lex Luthor'' difficult levels.
	\item Extended encoding on $9 \times 9$ puzzles of ``easy'', ``medium'', and
		``Lex Luthor'' difficulty levels.
	\item Extended encoding on unsolvable $9 \times 9$ puzzles of ``easy'',
		``medium'' and ``Lex Luthor'' difficult levels.
	\item Minimal encoding on $16 \times 16$ puzzles of
		``easy'', ``medium'', and ``Lex Luthor'' difficulty levels.
	\item Extended encoding	on $16 \times 16$ puzzles of
		``easy'', ``medium'', and ``Lex Luthor'' difficulty levels.
\end{enumerate}

We sample the results on a 3rd Gen. intel core i5-3337U cpu clocked at 1.80Ghz
with 4Gb of RAM.

\subsection{Data sample}
We are using an online board
generator\footnote{\url{http://sudoku-puzzles.merschat.com/}} with difficulties
ranging from ``easy'' to ``Lex Luthor''. Easier puzzles have many blocks
originally filled in, whereas puzzles at ``Lex Luthor'' difficulty only have a
few positions filled in, adding more potentially ambiguous cases. On a $9
\times 9$ board, an ``easy'' board may have more than 50\% of the squares
filled in, whereas a ``Lex Luthor'' board may have as few as 4 squares filled
in, which leads to more ambiguous cases and potentially many solutions for a
given puzzle.

\section{Results}
This section contains the results of the experiment.
The results are generated on a  3rd Gen. intel core i5-3337U cpu clocked at
1.80Ghz with 4Gb of RAM using ``minisat''. The resulting times only include the
reported time taken for the SAT solver to run, not the time for the encoder to
run on each puzzle.\\\\
\newpage
\subsection{Test 1: Minimal encoding on $9 \times 9$ puzzles of ``easy'', ``medium'',
and ``Lex Luthor'' difficulty levels.}
\FloatBarrier
\begin{table}[!h]
	\centering
	\begin{tabular}{c | l l l}
		Difficulty & Variables & Clauses & Parse time (s) \\\hline
		Easy 		& 729 & 1895 & 0.00 \\
		Medium 		& 729 & 1926 & 0.00 \\
		Lex Luthor 	& 729 & 7246 & 0.00
	\end{tabular}
	\caption{Minimal $9 \times 9$ Parse Information}
	\label{tab:minimal_9x9_parse}
\end{table}
\begin{table}[!h]
	\centering

	\begin{tabular}{c | l l p{20pt} p{20pt} l}
		Difficulty & Memory & Time (s) & Decisions &
		Conflicts & Propagations \\\hline
		Easy 		& 19.00 & 0 		& 1  & 0 & 729\\
		Medium 		& 19.00 & 0 		& 5  & 0 & 729\\
		Lex Luthor 	& 19.00 & 0.003333 	& 95 & 4 & 851
	\end{tabular}
	\caption{Minimal $9 \times 9$ Processing Information}
	\label{tab:minimal_9x9_proc}
\end{table}
\FloatBarrier

\subsection{Test 2: Minimal encoding on unsolvable $9 \times 9$ puzzles of ``easy'',
``medium'' and ``Lex Luthor'' difficult levels.}
\FloatBarrier
\begin{table}[!h]
	\centering
	\begin{tabular}{c | l l l}
		Difficulty & Variables &  Clauses & Parse time (s) \\\hline
		Easy 		& 729 & 1863 & 0.00 \\
		Medium 		& 729 & 1813 & 0.00 \\
		Lex Luthor 	& 729 & 6864 & 0.00
	\end{tabular}
	\caption{Unsolvable Minimal $9 \times 9$ Parse Information}
	\label{tab:minimal_9x9_unsolvable_parse}
\end{table}
\begin{table}[!h]
	\centering
	\begin{tabular}{c | l l p{20pt} p{20pt}  l }
		Difficulty & Memory & Time (s) & Decisions &
		Conflicts & Propagations \\\hline
		Easy 		& 19.00 & 0.003333 & 0  & 0  & 646 \\
		Medium 		& 19.00 & 0.003333 & 0  & 0  & 512 \\
		Lex Luthor 	& 19.00 & 0 	   & 86 & 53 & 2518
	\end{tabular}
	\caption{Unsolvable Minimal $9 \times 9$ Processing Information}
	\label{tab:minimal_9x9_unsolvable_proc}
\end{table}
\FloatBarrier

\subsection{Test 3: Extended encoding on $9 \times 9$ puzzles of ``easy'', ``medium'',
and ``Lex Luthor'' difficulty levels.}
\FloatBarrier
\begin{table}[!h]
	\centering
	\begin{tabular}{c | l l l}
		Difficulty & Variables & Clauses & Parse time (s) \\\hline
		Easy 		& 729 & 1895 & 0.00 \\
		Medium 		& 729 & 1953 & 0.00 \\
		Lex Luthor 	& 729 & 9217 & 0.00
	\end{tabular}
	\caption{Extended $9 \times 9$ Parse Information}
	\label{tab:extended_9x9_parse}
\end{table}
\begin{table}[!h]
	\centering
	\begin{tabular}{c | l l p{20pt} p{20pt} l}
		Difficulty & Memory & Time (s) & Decisions &
		Conflicts & Propagations \\\hline
		Easy 		& 19.00 & 0 	& 1  & 0 & 729\\
		Medium 		& 19.00 & 0 	& 2  & 0 & 729\\
		Lex Luthor 	& 19.00 & 0 	& 64 & 0 & 729
	\end{tabular}
	\caption{Extended $9 \times 9$ Processing Information}
	\label{tab:extended_9x9_proc}
\end{table}
\FloatBarrier

\subsection{Test 4: Extended encoding on unsolvable $9 \times 9$ puzzles of ``easy'',
``medium'' and ``Lex Luthor'' difficult levels.}
\FloatBarrier
\begin{table}[!h]
	\centering
	\begin{tabular}{c | l l l}
		Difficulty & Variables & Clauses & Parse time (s) \\\hline
		Easy 		& 729 & 1863 & 0.00 \\
		Medium 		& 729 & 1813 & 0.00 \\
		Lex Luthor 	& 729 & 8429 & 0.00
	\end{tabular}
	\caption{Unsolvable Extended $9 \times 9$ Parse Information}
	\label{tab:extended_9x9_unsolvable_parse}
\end{table}
\begin{table}[!h]
	\centering
	\begin{tabular}{c | l l p{20pt} p{20pt}  l }
		Difficulty & Memory & Time (s) & Decisions &
		Conflicts & Propagations \\\hline
		Easy 		& 19.00 & 0 & 0  & 0  & 646 \\
		Medium 		& 19.00 & 0.003333 & 0 & 0  & 512 \\
		Lex Luthor 	& 19.00 & 0.003333 & 0 & 0 & 252
	\end{tabular}
	\caption{Unsolvable Extended $9 \times 9$ Processing Information}
	\label{tab:extended_8x8_unsolvable_proc}
\end{table}
\FloatBarrier

\subsection{Test 5: Minimal encoding on $16 \times 16$ puzzles of ``easy'',
``medium'', and ``Lex Luthor'' difficulty levels.}
\FloatBarrier
\begin{table}[!h]
	\centering
	\begin{tabular}{c | l l l}
		Difficulty & Variables & Clauses & Parse time (s) \\\hline
		Easy		& 4096 & 8716  & 0.02\\
		Medium 		& 4096 & 30977 & 0.02\\
		Lex Luthor 	& 4096 & 73448 & 0.02\\
	\end{tabular}
	\caption{Minimal $16 \times 16$ Parse Information}
	\label{tab:minimal_16x16_parse}
\end{table}
\begin{table}[!h]
	\centering

	\begin{tabular}{c | l l p{20pt} p{20pt}  l }
		Difficulty & Memory & Time (s) & Decisions &
		Conflicts & Propagations \\\hline
		Easy 		& 20.00 & 0.016666 & 0  & 0  & 1367\\
		Medium 		& 20.00 & 0.026666 & 267 & 73 & 15880\\
		Lex Luthor 	& 21.00 & 0.036666 & 2446 & 272 & 27704
	\end{tabular}
	\caption{Minimal $16 \times 16$ Processing Information}
	\label{tab:minimal_16x16_proc}
\end{table}
\FloatBarrier

\subsection{Test 6: Extended encoding	on $16 \times 16$ puzzles of ``easy'',
``medium'', and ``Lex Luthor'' difficulty levels.}
\FloatBarrier
\begin{table}[!h]
	\centering
	\begin{tabular}{c | l l l}
		Difficulty & Variables & Clauses & Parse time (s) \\\hline
		Easy		& 4096 & 8761  & 0.02\\
		Medium 		& 4096 & 32205 & 0.03\\
		Lex Luthor 	& 4096 & 91595 & 0.04\\
	\end{tabular}
	\caption{Extended $16 \times 16$ Parse Information}
	\label{tab:extended_16x16_parse}
\end{table}
\begin{table}[!h]
	\centering

	\begin{tabular}{c | l l p{20pt} p{20pt} l }
		Difficulty & Memory & Time (s) & Decisions &
		Conflicts & Propagations \\\hline
		Easy 		& 20.00 & 0.019999 & 0 & 0 & 1367\\
		Medium 		& 20.00 & 0.033333 & 13 & 3 & 4214\\
		Lex Luthor 	& 23.00 & 0.029999 & 444 & 4 & 4371
	\end{tabular}
	\caption{Extended $16\times 16$ Processing Information}
	\label{tab:extended_16x16_proc}
\end{table}
\FloatBarrier

\newpage
\section{Discussion}
Our original hypothesis was that the minimal encoding would take less cpu time;
however, in all test cases, the number of propagations, decisions, and cpu time
decreased from the minimal encoding to the extended encoding.
\begin{table}[!h]
	\centering
	\begin{tabular}{cc | p{4pt} p{13pt} p{4pt} c | p{4pt} p{13pt} p{4pt} c}
		& & \multicolumn{4}{c | } {Minimal Encoding} & \multicolumn{4}
		{c} {Extended Encoding}\\
		size & level & vars & clauses & dec & prop & vars & clauses &
		dec & prop \\\hline
		$9\times9$ & easy & 	729 & 1895 & 1 & 729
		& 729 & 1895 & 1 & 729\\
		$9\times9$ & medium & 	729 & 1926 & 5 & 729
		& 729 & 1953 & 2 & 729 \\
		$9\times9$ & lex & 	729 & 7246 & 95 & 851
		& 729 & 9217 & 64& 729\\\hline

		$16\times16$ & easy &	4096 & 8716  & 0 & 1367 &
		4096 & 8761 & 0 & 1367\\
		$16\times16$ & medium &	4096 & 30977 & 267 & 15880 &
		4096 & 32205 & 13 & 4214\\
		$16\times16$ & lex & 	4096 & 73448 & 2446 & 27704 &
		4096 & 91595 & 444 & 4371\\

	\end{tabular}
	\caption{Minimal encoding versus Extended encoding on solvable puzzles}
	\label{tab:summary_solvable}
\end{table}
\begin{table}[!h]
	\centering
	\begin{tabular}{cc | p{4pt} p{10pt} p{4pt} c | p{4pt} p{10pt} p{4pt} c}
		& & \multicolumn{4}{c | } {Minimal Encoding} & \multicolumn{4}
		{c} {Extended Encoding}\\
		size & level & vars & clauses & dec & prop & vars & clauses &
		dec & prop \\\hline
		$9\times9$ & easy 	& 729 & 1863 & 0  & 646 &
		729 & 1863 & 0 & 646 \\
		$9\times9$ & medium 	& 729 & 1813 & 0  & 512 &
		729 & 1813 & 0 & 512 \\
		$9\times9$ & lex 	& 729 & 6864 & 86 & 2518 &
		729 & 8429 & 0 & 252 \\
	\end{tabular}
	\caption{Minimal encoding versus Extended encoding on unsolvable puzzles}
	\label{tab:summary_unsolvable}
\end{table}
\FloatBarrier

In table \ref{tab:summary_solvable}, we can see that the growth of the
decisions and propagations over difficulty levels is less given an extended
encoding of the board over a minimal encoding of the board. This trend holds
true as the size of the puzzle grows.

In table \ref{tab:summary_unsolvable}, we can see that both the extended
encoding and the minimal encoding ran in the same time for unsolvable puzzles
where there were no possible ambiguities. The ``Lex Luthor'' puzzle does not
necessarily have a single solution so we see deviation between the two methods.
In this case, we can see that the extended encoding was able to arrive at the
conclusion that the board was unsatisfiable in fewer decisions and
propagations.

\subsection{Limitations}
Theoretically our program can use any $n \times n$ puzzle, so long a $n$ is a
perfect square; however, due to the limitations of encoding numbers
larger than $35$, our program can only take puzzles up to $25 \times 25$. A
possible method of extending the usage of the solver is to allow a delimiter
between values in each cell.

\section{Future Work}
As stated\cite{Kwon2005}, the growth of the extended encoding is $O(n^4)$,
which may cause the total runtime of the extended encoding to be larger than
the minimal encoding simply due to the encoding process. It would be
interesting to further the study on $81\times 81$ puzzles and $100 \times 100$
puzzles to see at what point the extended encoding is no-longer feasible due to
the encoding runtime. Furthermore, it would be interesting to investigate the
efficient encoding and compare that against the extended and minimal encodings.

\section{Threats to Validity}
In the paper presented by Lynce it states that the extended encoding
requires two extra properties: that the puzzles only have one solution and
that puzzles can be solved with only reasoning. The boards we were testing
with, specifically the ``Lex Luther'' difficulty are not guaranteed to maintain
these extra properties. From visual inspection, the minimal and extended
encodings were able to correctly generate two correct solutions for the same
puzzle, but decreasing the accuracy of the comparison.

Our original hypothesis was using cpu time as a measurement of difficulty , but
we found that the number of propagations and decisions was a better measurement
of the difficulty of a given puzzle.

\section{Conclusion}
We found that extended encoding ran with fewer propagations, decisions, and cpu
time than minimal encoding, but with higher memory usage and more clauses. The
growth of the clauses may limit the size of boards that the extended encoding
scheme is feasible for encoding.

Our program is able to run on $n \times n$ puzzles up to $n = 25$ where $n$ is
a perfect square.

\bibliographystyle{plain}
\bibliography{references}

\balance

\end{document}
